\documentclass[12pt]{article}
\usepackage[english,brazil]{babel}

\title{Relat\'orio - PSG2}
\author{Bruno Yoshikazu Shimada}
\date{Novembro 2017}

\begin{document}
\maketitle

\newpage
\section*{\centering{Resumo}}

Vivemos em uma \'epoca que a cada dia, novas solu\c{c}\~oes digitais s\~ao lan\c{c}adas para resolver nossos problemas cotidianos de uma maneira simplificada que se incorporam ao nosso cotidiano de uma maneira que ap\'os algum tempo n\~ao conseguimos imaginar como n\'os conseguimos viver sem isso por tanto tempo.

Entre as solu\c{c}\~oes digitais podemos citar por exemplo o ramo econ\^omico/banc\'ario, com desde os b\'asicos aplicativos de bancos para consulta a extratos, pagamentos de boletos, etc, at\'e solu'\c{c}\~oes mais sofisticadas como aplicativos para receber e processar pagamentos por cart\~ao.

Por\'em um problema corriqueiro que temos no dia-a-dia \'e o pagamento de pequenas transa\c{c}\~oes monet\'arias, como tomar um caf\'e na padaria, comprar um chiclete na bomboniere, emprestar uma pequena quantia de dinheiro para um conhecido, entre outras que na maioria das vezes n\~ao passam de 10R\$ e que geram uma perda de tempo de ter que realizar o pagamento com um cart\~ao, digitando a senha pessoal em uma m\'aquina e torcendo para o servi\c{c}o da operadora do cart\~ao estar disponível no momento.

Para resolver esse problema entram em cena os servi\c{c}os de micropagamento, que buscam solucionar o problema de fazer um pagamento ou uma transa\c{c}\~ao de pequeno valor entre duas pessoas, independente de ser pessoa-a-pessoa ou pessoa-a-neg\'ocio, de forma r\'apida, eficaz, segura e descomplicada. O prop\'osito deste trabalho \'e ent\~ao desenvolver um aplicativo Android, SO mais comum entre os usu\'arios de celular no Brasil, que gerencie as transações entre usu\'arios do aplicativo e que fa\c{c}a valer as caracter\'isticas descritas anteriormente.
\newline



\textbf{Palavras Chaves}
\begin{itemize}
	\item Micropagamentos
	\item \textit{Android}
	\item \textit{Fintech}
\end{itemize}
\newpage
\selectlanguage{english}
\begin{abstract}
We live in time that everyday new digital solutions are released to solve our daily problems in a simplified manner, that merges into our days that after some time we are unable to imagine how we lived so much time without it.

Among these digital solutions we can mention for example the economic/banking areas, that has the basic bank app for checking account information, payment of tickets, etc.. To more sophisticated apps that receive and process payments by card.

However a common problem we have in our daily lives is the payment of small monetary transactions, such as having a coffee at a bakery, buying gum from a local store, lending money to an acquaintance, among others that in most times don’t surpass 10 BRL and generate a loss of time for having to make the payment with a card, typing our personal password on a machine and hoping the service of the card operator will be available at the moment.

To solve this problem the micropayments services come into play, seeking to solve the problem of making the payment or transaction of a small amount of money between two persons, regardless of being a person-to-person or person-to-business, in a quick, efficient, safe and uncomplicated way. The purpose of this work is to develop an Android app, the most common OS among mobile users in Brazil, that manage transaction between users of the app and enforce the characteristic described before.

\end{abstract}	
\end{document}